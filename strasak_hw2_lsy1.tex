\documentclass[10pt]{article}
\usepackage[utf8]{inputenc}

% Standard mathematical packages
\usepackage{amsmath}
\usepackage{amssymb}
% The bmatrix environment used for matrices is provided by amsmath.

\setlength{\voffset}{-0.65in}   % Upper border
\setlength{\hoffset}{-0.65in}   % Left border

\title{[B3M35LSY] Assignment}
\author{David Strašák}
\date{05.10.2025}

\begin{document}
	\maketitle
	
	{\bf Exercises for lecture 2}
	
	\begin{enumerate}
		
		\item \textbf{For the system $\dot{x}(t) = Ax(t)$ with
		\begin{eqnarray*}
			A=\begin{bmatrix}
				0 & 1\\
				1 & 0 
			\end{bmatrix}
		\end{eqnarray*}
		determine all initial conditions $x(0)$ such that any of the elements of $x(t) \rightarrow \infty$ as $t \rightarrow \infty$.}
		
		In other words, we want to see for which initial condition the value of $\mathbf{x}(t)$ diverges.
		
		The eigenvalues for this system are $\lambda_1 = 1$ and $\lambda_2 = -1$.
		The corresponding eigenvectors are $\mathbf{v}_1 = \begin{bmatrix} 1 \\ 1 \end{bmatrix}$ and $\mathbf{v}_2 = \begin{bmatrix} 1 \\ -1 \end{bmatrix}$.
		This gives us the Jordan form matrix $J = \begin{bmatrix} 1 & 0 \\ 0 & -1 \end{bmatrix}$ and the transformation matrix $P = \begin{bmatrix} 1 & 1 \\ 1 & -1 \end{bmatrix}$.
		
		We can then calculate the value of $\mathbf{x}(t)$ through the formula:
		$$
		\mathbf{x}(t) = P e^{Jt} P^{-1} \mathbf{x}(0)
		$$
		Assuming the initial condition $\mathbf{x}(0) = \begin{bmatrix} c_1 \\ c_2 \end{bmatrix}$, the full expression is:
		$$
		\begin{bmatrix} x_1(t) \\ x_2(t) \end{bmatrix} = \begin{bmatrix} 1 & 1 \\ 1 & -1 \end{bmatrix} \begin{bmatrix} e^t & 0 \\ 0 & e^{-t} \end{bmatrix} \begin{bmatrix} 1/2 & 1/2 \\ 1/2 & -1/2 \end{bmatrix} \begin{bmatrix} c_1 \\ c_2 \end{bmatrix}
		$$
		
		Modified it is:
		$$
		\begin{bmatrix} x_1(t) \\ x_2(t) \end{bmatrix} = \begin{bmatrix} e^t & e^{-t} \\ e^t & -e^{-t} \end{bmatrix} \begin{bmatrix} k_1 \\ k_2 \end{bmatrix}
		$$
		where $k_1 = \frac{1}{2}(c_1+c_2)$ and $k_2=\frac{1}{2}(c_1-c_2)$. The values $c_1$ and $c_2$ are defined by $\mathbf{x}(0) = \begin{bmatrix} c_1 \\ c_2 \end{bmatrix}$.
		
		The components of $\mathbf{x}(t)$ are:
		\begin{align*}
			x_1(t) &= k_1 e^t + k_2 e^{-t} \\
			x_2(t) &= k_1 e^t - k_2 e^{-t}
		\end{align*}
		
		Since $e^{-t} \rightarrow 0$ as $t \rightarrow \infty$, the divergence is determined by the term $e^t$. 
		
		The term $k_1 e^t$ will diverge to $\pm\infty$ if $k_1$ is not zero. 
		
		So the final solution is:
		$$ \frac{1}{2}(c_1+c_2) \neq 0 $$
		This system will diverge whenever $c_1+c_2$ is not $0$.
		
		\item \textbf{For the system $x(k+1) = Ax(k) + Bu(k)$ with 
			\begin{eqnarray*}
				A=\begin{bmatrix}
					2 & 0\\
					0 & -1
				\end{bmatrix},\,
				B=\begin{bmatrix}
					1\\
					2
				\end{bmatrix}
			\end{eqnarray*}
			determine all initial conditions $x(0)$ such that there exists an input sequence $u(k)$ so that the state remains  at $x(0)$, i. e. $x(k) = x(0)$ for all $k \geq 0$.}
		
		If $\mathbf{x}(k+1) = \mathbf{x}(k) = \mathbf{x}(0)$, the equation becomes $\mathbf{x}(0) = A\mathbf{x}(0) + B u(0)$. Let $\mathbf{x}(0) = \begin{bmatrix} x_1 \\ x_2 \end{bmatrix}$ and $u(0)=u$.
		We can write out the equation in matrix form:
		$$
		\begin{bmatrix} x_1 \\ x_2 \end{bmatrix} = \begin{bmatrix} 2 & 0 \\ 0 & -1 \end{bmatrix} \begin{bmatrix} x_1 \\ x_2 \end{bmatrix} + \begin{bmatrix} 1 \\ 2 \end{bmatrix} u
		$$
		This gives us the following linear system of equations:
		\begin{align*}
			x_1 &= 2x_1 + u \implies -x_1 = u \\
			x_2 &= -x_2 + 2u \implies 2x_2 = 2u \implies x_2 = u
		\end{align*}
		That gives us the relation between $x_1$ and $x_2$:
		$$ x_2 = -x_1 $$
		In order for the input $u$ to exist, the initial condition $\mathbf{x}(0)$ must satisfy this linear relationship.
		
		So the initial conditions are of the form $\mathbf{x}(0) = \begin{bmatrix} k \\ -k \end{bmatrix}$ where $k$ can be any real number ($k \in \mathbb{R}$).
		
		\item \textbf{Consider continuous-time system $\dot{x}(t) = Ax(t) + Bu(t),\,y(t) = Cx(t)$ with 
			\begin{eqnarray*}
				A=\begin{bmatrix}
					-1 & 1\\
					-1 & 0
				\end{bmatrix},\,
				B=\begin{bmatrix}
					0\\
					1
				\end{bmatrix},\,
				C = [1 & 0].
			\end{eqnarray*}
			Choose $\alpha$ and $x(0)$ such that $u(t)={\rm e}^{\alpha t}$ produces $y(t) = {\rm e}^{\alpha t}$, i. e. the system behaves as a unity gain.
		}
		
		From the output equation $y(t) = C\mathbf{x}(t)$, and the desired output $y(t) = e^{\alpha t}$, we immediately have the function for $x_1(t)$:
		$$ x_1(t) = e^{\alpha t} $$
		We can take the derivative of this function and substitute it into the first line of the state equation ($\dot{x}_1 = -x_1 + x_2$):
		$$ \dot{x}_1(t) = \alpha e^{\alpha t} $$
		Substituting gives:
		$$ \alpha e^{\alpha t} = -e^{\alpha t} + x_2(t) $$
		From this we can find the definition for $x_2(t)$:
		$$ x_2(t) = \alpha e^{\alpha t} + e^{\alpha t} = (\alpha + 1)e^{\alpha t} $$
		Now that we have the definitions of $x_1(t)$ and $x_2(t)$, we can find the initial condition (though we cannot specify it just yet because we don't know $\alpha$):
		$$ \mathbf{x}(0) = \begin{bmatrix} x_1(0) \\ x_2(0) \end{bmatrix} = \begin{bmatrix} 1 \\ \alpha + 1 \end{bmatrix} $$
		We can once again make a derivative of $x_2(t)$ and put it inside the second state equation ($\dot{x}_2 = -x_1 + u(t)$):
		$$ \dot{x}_2(t) = \frac{d}{dt} [(\alpha + 1)e^{\alpha t}] = \alpha(\alpha + 1)e^{\alpha t} $$
		Substituting all functions into the second state equation:
		$$
		\alpha(\alpha + 1)e^{\alpha t} = -e^{\alpha t} + e^{\alpha t} = 0
		$$
		Since $e^{\alpha t} \neq 0$, we must have $\alpha(\alpha + 1) = 0$.
		Now we can find the value of $\alpha$:
		$$ \alpha(\alpha + 1) = 0 $$
		Alpha is either $\alpha = 0$ or $\alpha = -1$.
		
		We will choose $\alpha = 0$. Now we can find the initial condition:
		$$ \mathbf{x}(0) = \begin{bmatrix} 1 \\ 0 + 1 \end{bmatrix} = \begin{bmatrix} 1 \\ 1 \end{bmatrix} $$
		
		\item \textbf{For a continuous-time system described by
			$\dot{x}(t) = Ax(t)$ with
			\begin{eqnarray*}
				A=\begin{bmatrix}
					-1 & \alpha & 2\\
					0 & -2 & -1\\
					0 & 1 & 0
				\end{bmatrix}
			\end{eqnarray*}
			determine all the values of $\alpha \in \Re$ such that the system
			contains as many modes as possible.
		}
		
		Since the matrix is $3 \times 3$, the system has a maximum number of three modes. The number of modes will be given by subtracting the geometric multiplicity from the algebraic multiplicity.
		
		The eigenvalue of this system is not influenced by the $\alpha$ and it is $\lambda = 1$. If we calculate $A - \lambda I$ we will get the matrix:
		$$
		A - \lambda I = \begin{bmatrix}
			0 & \alpha & 2\\
			0 & -1 & -1\\
			0 & 1 & 1
		\end{bmatrix}
		$$
		The algebraic multiplicity of the eigenvalue is $3$.
		If we were to choose $\alpha = 2$ then the geometric multiplicity would be $2$. If we were to choose any other number than $2$, the geometric multiplicity would be 1.That means, to have the highest amount of modes ($2$ modes), $\alpha$ cannot be $2$.
		
		It is possible to find an eigenvectors and two generalized eigenvectors, so the $J$ matrix will look like:
		$$
		J = \begin{bmatrix}
			-1 & 1 & 0\\
			0 & -1 & 1\\
			0 & 0 & -1
		\end{bmatrix}
		$$
		
		\item \textbf{A certain man put a pair of rabbits in a place surrounded on all sides by a wall.
			What will be the number of pairs of rabbits in a month $k$ if it is supposed that every month each pair begs a new pair which from the second month
			on becomes productive? Suppose that the rabbits do not die. Write corresponding state-space equations and solve them in explicit form (i. e. as a scalar function of $k$).
		}
		
		We will model this as a system with two state variables. $x_1(k)$ is the number of productive pairs and $x_2(k)$ is the number of young pairs. The output $y(k)$ will be the total number of rabbits.
		
		The initial condition is $\mathbf{x}(0) = \begin{bmatrix} 1 \\ 0 \end{bmatrix}$ because there is one initial productive pair.
		
		The state update equations are:
		\begin{align*}
			x_1(k+1) &= x_1(k) + x_2(k) \\
			x_2(k+1) &= x_1(k)
		\end{align*}
		The corresponding discrete-time state-space equation is:
		$$
		\mathbf{x}(k+1) = A\mathbf{x}(k)
		$$
		so the state matrix is:
		$$
		A=\begin{bmatrix}
			1 & 1\\
			1 & 0
		\end{bmatrix}
		$$
		The input can be $\mathbf{u}(k)=0$, because we are not inputting any new rabbits into the system. Because of this the matrix B and D can be zero.
		The output $y(k)$ is:
		$$
		y(k) = C\mathbf{x}(k)
		$$
		where $C = \begin{bmatrix} 1 & 1 \end{bmatrix}$.
		
		The explicit solution for the state at time $k$ is given by:
		$$
		\mathbf{x}(k) = A^k \mathbf{x}(0)
		$$
		
	\end{enumerate}
	
\end{document}
