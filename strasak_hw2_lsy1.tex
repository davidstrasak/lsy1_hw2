\documentclass[10pt]{article} % use larger type; default would be 10pt

\usepackage[utf8]{inputenc} % set input encoding (not needed with XeLaTeX)

\setlength{\voffset}{-0.65in}   %upper border 
\setlength{\hoffset}{-0.65in}   %left border

\title{[B3M35LSY] Assignment}
\author{David Strašák}
\date{05.10.2025} % Activate to display a given date or no date (if empty),
         % otherwise the current date is printed

\usepackage{amsmath}
\usepackage{amssymb}

\begin{document}
%\maketitle

{\bf Exercises for lecture 2}

% ========= 2nd WEEK 25/26 ================================
\begin{enumerate}
\item For the system $\dot{x}(t) = Ax(t)$ with
\begin{eqnarray*}
A=\begin{bmatrix}
0 & 1\\
1 & 0 
\end{bmatrix}
\end{eqnarray*}
determine all initial conditions $x(0)$ such that any of the elements of $x(t) \rightarrow \infty$ as $t \rightarrow \infty$.

In other words we want to see for which initial condition the value of x diverges.

Eigenvalues for this system are 1 and -1.
Eigenvector for the first eigenvalue is [1;1]
Eigenvector for the second eigenvalue is [1;-1]
That gives us the matrix J = [1,0;0,-1] and the matrix P = [1,1;1,-1]
We can then calculate the value of x through this formula:
x = P*e^(Jt)*p^(-1)*x0
That is:
[x1;x2] = [1,1;1,-1]*[e^t,0;0,e^(-t)]*[1/2,1/2;1/2,-1/2]*[c1;c2]
Modified it is:
[x1;x2] = [e^t,e^(-t);e^t,-e^(-t)]*[k1;k2]
where k1 = 1/2*(c1+c2) and k2=1/2*(c1-c2)
c1 and c2 are defined here: x0 = [c1;c2]
e^(-t) always converges into 0, so we will focus on the rows which are e^t*k1.

e^t*k1 will always converge if k1 is not zero. So the final solution is:

1/2*(c1+c2) is not 0.

This system will diverge whenever c1+c2 is not 0.

\item For the system $x(k+1) = Ax(k) + Bu(k)$ with 
\begin{eqnarray*}
A=\begin{bmatrix}
2 & 0\\
0 & -1
\end{bmatrix},\,
B=\begin{bmatrix}
1\\
2
\end{bmatrix}
\end{eqnarray*}
determine all initial conditions $x(0)$ such that there exists an input sequence $u(k)$ so that the state remains  at $x(0)$, i. e. $x(k) = x(0)$ for all $k \geq 0$.

We can write out the equation in matrix form:
[x1(k+1);x2(k+1)] = [2,0;0,-1]*[x1(k);x2(k)] + [1;2]*u(k)
Since x(k+1) = x(k) = x(0) we can write this out into this linear system of equations:
-x1 = u
x2 = u
That gives us the relation between x1 and x2:
x2 = -x1

In order for the input u to exist, the x1 and x2 points need to be lying on this line x2 = -x1.

So the input is x0 = [k, -k] where k can be any real number.

\item Consider continuous-time system $\dot{x}(t) = Ax(t) + Bu(t),\,y(t) = Cx(t)$ with 
\begin{eqnarray*}
A=\begin{bmatrix}
-1 & 1\\
-1 & 0
\end{bmatrix},\,
B=\begin{bmatrix}
0\\
1
\end{bmatrix},\,
C = [1 & 0].
\end{eqnarray*}
Choose $\alpha$ and $x(0)$ such that $u(t)={\rm e}^{\alpha t}$ produces $y(t) = {\rm e}^{\alpha t}$, i. e. the system behaves as a unity gain.

We can write this system of equations into this form:
[x1'; x2'] = [-1,1;-1,0]*[x1;x2] + [0;e^(alpha*t)]
And because we got this from the line with the matrix C: x1 = e^(alpha*t), we immediately have the function of x1.
We can make a derivative of this function and put substitute it into the first line of the equation with matrixes A and B.
x1' = alpha*e^(alpha*t)
alpha*e^(alpha*t) = -e^(alpha*t) + x2(t)
From this we can find the definition for x2(t)
x2(t) = alpha*e^(alpha*t)+e^(alpha*t)
Now that we have the definitions of x1 and x2 we can almost find the initial conditions of this equation:
x(0) = [1; alpha + 1]
We can once again make a derivative of x2 and put it inside the second equation with the matrixes A and B.
x2' = alpha^2*e^(alpha*t) + alpha*e^(alpha*t)
alpha^2*e^(alpha*t) + alpha*e^(alpha*t) = -e^(alpha*t) + e^(alpha*t) = 0
Now we can find the value of alpha:
alpha*(alpha + 1)*e^(alpha*t) = 0
Alpha is either 0 or -1.
We will choose alpha to be 0. Now we can find the initial conditions of the equation:
x0 = [1;0 + 1]
For alpha = 0.

\item For a continuous-time system described by
$\dot{x}(t) = Ax(t)$ with
\begin{eqnarray*}
A=\begin{bmatrix}
-1 & \alpha & 2\\
0 & -2 & -1\\
0 & 1 & 0
\end{bmatrix}
\end{eqnarray*}
determine all the values of $\alpha \in \Re$ such that the system
contains as many modes as possible.

Since the matrix is 3x3, the system has a maximum number of three modes. The number of modes will be given by subtracting the geometric multiplicity from the algebraic multiplicity.

The eigenvalue of this system is not influenced by the alpha and it is lambda = 1. If we calculate A - lambda*I we will get the matrix:
A - lambda*I = [0, alpha, 2; 0, -1, -1; 0, 1, 1]
Algebraic multiplicity of the eigenvalue is 3.
If we were to choose alpha = 2 then the geometric multiplicity would be 2. That means, to have the highest amount of modes (2 modes), alpha cannot be 2.

It is possible to find an eigenvectors and two generalized eigenvectors, so the J matrix will look like:
J = [-1, 1, 0; 0, -1, 1; 0, 0, -1]

\item A certain man put a pair of rabbits in a place surrounded on all sides by a wall.
What will be the number of pairs of rabbits in a month $k$ if it is supposed that every month each pair begs a new pair which from the second month
on becomes productive? Suppose that the rabbits do not die. Write corresponding state-space equations and solve them in explicit form (i. e. as a scalar function of $k$).

I would like to model this system as a descrete system where the state variable x is only a scalar - the number of rabbits.

The initial condition for the number of rabbits will be 1 pair or it can be 2 rabbits.

The matrix A is in this case also only a scalar (because the state is a scalar) and it will be A = 2 because the number of rabbits multiplies every month.

The input u equals to zero because no rabbits are inputted into the system every month. Thus the matrix (also a scalar in this case) B can be any value - we can keep it at B = 1.

Finally the output y is the number of rabbits, which means that the matrix C is 1. The matrix D can also be any number since there are no inputs u, but we will keep it at 1, because the number of rabbits added directly adds to the total number of rabbits.

That gives us these equations which describe the system:
x(k+1) = A*x(k)
y(k) = 1*x(k)

These equations can be rewritten into this form:
x(k) = A^(k-k0)*x0

Where x0 is the initial number of rabbits and k0 is the starting month.

\end{enumerate}

\end{document}
